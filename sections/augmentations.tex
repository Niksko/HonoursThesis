\section{Augmentations and modifications of the IIM}

\subsection{The Explicit-Jump Immersed Interface Method (EJIIM)}

The Explicit-Jump Immersed Interface Method (herein EJIIM) has been develop by Wiegmann \cite{wiegmann98} and Wiegmann and Bube \cite{wiegmannbube98}, \cite{wiegmannbube00}.

Instead of generating new finite difference coefficients (the $\gamma$s found above), the EJIIM derives correction terms for the existing finite difference methods using Taylor series and the known jumps over the interfaces.
This is advantageous as it avoids having to solve a linear system for the $\gamma$s and it also allows for higher order finite difference schemes to be generated by creating higher order correction terms.
However this method can only be employed if the jump conditions are known explicitly.

Recently, Rutka \cite{rutka08} and Rutka and Li \cite{rutkali08} have used the EJIIM to solve the Navier-Stokes equations for two fluids with a singular force term at the boundary.
The EJIIM is particularly suited to this problem as standard augmented methods (which are required due to a coupling between jump conditions for the pressure and velocity over the interface) are difficult to implement due to the challenge in finding augmented variables and equations.

\subsection{The Matched Interface and Boundary method (MIB)}

The Matched Interface and Boundary (herein MIB) method has been developed in \cite{zhaowei04}, \cite{zhouwei06} and \cite{zhouetal06}.
The MIB method treats points either side of the interface as being in separate domains, introducing fictitious points where necessary to complete the finite difference stencil.
These fictitious points are then coupled with their counterparts on the other side of the interface and the jump conditions.
This allows for the fictitious points to be solved for and then the finite difference stencils can be completed.
Since we now have accurate estimates on the fictitious points, we can repeat this algorithm to generate further points and thus increase the accuracy of our final solution by using a larger finite difference stencil.
This is a great advantage over the standard IIM because it implies that increased accuracy can be obtained via an algorithm instead of via manual computation as with the IIM.

In practice, Zhou et. al. \cite{zhouetal06} have achieved 16th order accuracy when the interfaces are straight lines, and 6th order accuracy for elliptic problems with curved interfaces.

\subsection{Immersed Finite Element Methods (IFEMs)}

The IIM using a finite element (FE) formulation has been developed in \cite{li98b}, \cite{lilinwu03} and \cite{kafafyetal05}.
For standard FE methods, second order accuracy is guaranteed if the interface lies on a grid-point.
However in higher dimensions it may be difficult or costly to ensure this, especially when the interface is moving.
The IFEM attempts to retain this accuracy while allowing for a uniform grid that does not have to conform to interfaces.

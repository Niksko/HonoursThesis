\section*{Introduction}
\addcontentsline{toc}{section}{Introduction}
The Immersed Interface Method (IIM) is a numerical technique for solving differential equations where either the solution, derivatives of the solution or parameters in the equation have discontinuities over the domain.
It is fundamentally a sharp interface method, meaning that it attempts to preserve the sharpness of discontinuities instead of smearing them as with other methods such as Peskin's Immersed Boundary Method.
Since its inception it has proven to be a robust method that is easily modified and adapted to solve a wide variety of problems in the physical and natural sciences. 
It has also been extensively improved, adapted and combined with existing methods to solve problems that would be otherwise difficult to solve due to the inability of standard methods to deal with discontinuities.

This thesis will begin with an overview of the immersed interface method including some applications, before discussing and comparing three different methods for solving time dependent PDEs in one spatial dimension.

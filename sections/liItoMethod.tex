\section{The Li and Ito method} \label{liIto}
The first method we will consider is presented in Li and Ito \cite{liito06}.
We will illustrate this method by demonstrating its derivation using the heat equation and a forward time centered space differencing scheme, but it is equally applicable to other equations and differencing schemes.

We wish to solve the equation
\begin{equation}
    \frac{\partial u}{\partial t} = \kappa \frac{\partial^2 u}{\partial x^2}
\end{equation}

on $0 \leq x \leq 1$, $t > 0$ and with given initial and boundary conditions

\begin{equation}
    u(x,0) = u_0(x), \qquad u(0,t) = a, \qquad u(1,t) = b,
\end{equation}

with jump conditions

\begin{equation}
    \left[u\right]_\alpha := u^+(\alpha) - u^-(\alpha) = w(t), \qquad
    \left[\frac{\partial u}{\partial x}\right]_\alpha := \frac{\partial u}{\partial x}^+(\alpha) - \frac{\partial u}{\partial x}^-(\alpha) = v(t)
\end{equation}

at some moving interface
\begin{equation}
    x = \alpha(t).
\end{equation}

We can discretize this by choosing small stepsizes $\Delta x$ and $\Delta t$ and letting
\begin{align}
    x_i &= i \Delta x, \quad i = 0, 1, \ldots, N \quad x_N = 1 \\
    t_k &= k \Delta t, \quad k = 0, 1, \ldots
\end{align}

Now we approximate the differential equation using forward differences in time and centered differences in space.
As with the IIM for ODEs, we will use a standard discretisation scheme at regular points and a modified scheme near the interface.
We will use the FTCS method in these overviews of the interface correction methods to illustrate the point.

The difference equation we obtain for regular points is

\begin{equation}
    \frac{u_i^{k+1} - u_i^k}{\Delta t} = \kappa \frac{u_{i-1}^k - 2 u_i^k + u_{i+1}^k}{(\Delta x)^2}
\end{equation}

where $u_i^k$ is an approximation to $u(x_i,t_k)$.

This gives that
\begin{align}
    u_i^{k+1} &= u_i^k + \frac{\kappa \Delta t}{\Delta x^2} \left(u_{i-1}^k - 2 u_i^k + u_{i+1}^k\right) \\
    \Rightarrow \quad u_i^{k+1} &= u_{i-1}^k \frac{\kappa \Delta t}{\Delta x^2} + u_i^k \left(1 - 2 \frac{\kappa \Delta t}{\Delta x^2}\right) + u_{i+1}^k \frac{\kappa \Delta t}{\Delta x^2}
\end{align}

At irregular points, we can use the standard IIM method to discretise $\partial^2 u / \partial x^2$ but we require a different method to ensure that the discretisation for $\partial u / \partial t$ is second order accurate.
Reduction of accuracy can occur when the interface crosses the grid line $x=x_i$ at some time $t=\tau$, $t_k < \tau < t_{k+1}$.
This may result in a discontinuity in $\partial u / \partial t$ which means that

\begin{equation}
    \lim_{t \to \tau^+} \frac{\partial u}{\partial t}(t) \neq \lim_{t \to \tau^-} \frac{\partial u}{\partial t}(t)
\end{equation}

In order to correct for this we add a term to the approximation to $\partial u / \partial t$ which is derived in a similar way to the derivation of the IIM in section \ref{OneD}.

We begin by expanding $u(x_i,t_{k+1})$ and $u(x_i,t_k)$ about the grid crossing time $t=\tau$.

\begin{align}
    u(x_i,t_k) &= u^-(x_i,\tau) + (t_k - \tau)\frac{\partial u}{\partial t}^-(x_i,\tau) + \mathcal{O}(\Delta t^2) \\
    u(x_i,t_{k+1}) &= u^+(x_i,\tau) + (t_{k+1} - \tau)\frac{\partial u}{\partial t}^+(x_i,\tau) + \mathcal{O}(\Delta t^2) \label{liItoRightInterface}
\end{align}

Using the jump conditions we can write equation \ref{liItoRightInterface} as
\begin{equation}
    u(x_i,t_{k+1}) = u^-(x_i,\tau) + [u]_\tau + (t_{k+1} - \tau)\frac{\partial u}{\partial t}^-(x_i,\tau) + (t_{k+1} - \tau) \left[\frac{\partial u}{\partial t}\right]_\tau +  \mathcal{O}(\Delta t^2)
\end{equation}

Combining these two expressions gives
\begin{equation}
    u(x_i,t_{k+1}) - u(x_i,t_k) = \Delta t \frac{\partial u}{\partial t}^-(x_i,\tau) + [u]_\tau + (t_{k+1} - \tau) \left[\frac{\partial u}{\partial t}\right]_\tau + \mathcal{O}(\Delta t^2)
\end{equation}

Hence,
\begin{equation}
     \frac{\partial u}{\partial t}^-(x_i,\tau) =  \frac{u(x_i,t_{k+1}) - u(x_i,t_k)}{\Delta t} - \frac{[u]_\tau}{\Delta t} - \frac{(t_{k+1} - \tau)}{\Delta t} \left[\frac{\partial u}{\partial t}\right]_\tau + \mathcal{O}(\Delta t)
\end{equation}

We can determine $[u]_\tau$ from the jump conditions by setting $\alpha = \alpha(\tau)$, but $[\partial u / \partial t]_\tau$ requires some derivation.

\begin{equation}
    u^+(\alpha,t) - u^-(\alpha,t) = w(t)
\end{equation}

Differentiating with respect to $t$ gives

\begin{equation}
    \left(\frac{\partial u}{\partial x}^+(\alpha,t) - \frac{\partial u}{\partial x}^-(\alpha,t)\right)\frac{d \alpha}{d t} + \frac{\partial u}{\partial t}^+(\alpha,t) - \frac{\partial u}{\partial t}^-(\alpha,t) = w'(t)
\end{equation}

Hence

\begin{equation}
    \left[\frac{\partial u}{\partial t}\right] = w'(t) - \left[\frac{\partial u}{\partial x}\right] \frac{d \alpha}{d t}
\end{equation}

We can now solve the system using standard techniques.

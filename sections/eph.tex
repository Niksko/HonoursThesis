\section{The IIM for elliptic, parabolic and hyperbolic interface problems}
\subsection{Elliptic problems}
\label{ellipticProblems}

Much of the analysis in section \ref{OneD} extends to elliptic problems in 2D such as Poisson's equation
\begin{equation}
    \nabla^2 u = f.
\end{equation}
However there are some differences when generalizing to two and higher dimensions.
\begin{itemize}
    \item Irregular points are now defined to be any points whose finite difference stencil is crossed by the interface.
    \item A point $(x_{i}^*,y_{i}^*)$ must be chosen chosen on the interface that is nearby the irregular point about which to expand the Taylor series.
    \item A coordinate transformation into coordinates normal and tangential to $(x_{i}^*,y_{i}^*)$ simplifies the interface relations.
    \item In addition to the normal finite difference stencil, an additional point is often required to ensure that the system for $\gamma$ is non-singular. In \cite{fogelsonkeener00} the authors present a method for choosing the additional point if necessary based on ensuring the diagonal dominance of the resulting system for $\gamma$.
\end{itemize}

Jump conditions are then derived based on these new coordinates.
In 2D, these jump conditions usually take the form of

\begin{align}
    [u]_\Gamma & =  w(s), \\ \nonumber
    \left[\frac{\partial u}{\partial n}\right]_\Gamma & = v(s) ,
\end{align}

where $\Gamma$ is a closed curve in the domain that defines the position of the interface, $n$ is the unit normal, $s$ is the arc length parameter along the interface $\Gamma$ and $v(s)$ and $w(s)$ define the jump conditions which we assume to be specified or otherwise determined from the physical properties of the problem.

Once these have been determined, a system of equations can be generated to solve for the unknown finite difference coefficients at each irregular point.
These coefficients can then be inserted into the larger system and the interface problem can be solved.

Some added complexity is inherent when solving elliptic equations due to the requirement of solving a large linear system accurately and quickly.
If the parameters in the elliptic equation are piece-wise constant, then a fast method exists for solving the system \cite{li98}.
This method involves solving a modified system which is easier to solve, and has the useful property that the number of iterations required for convergence is not related to the mesh size or the size of the jump in the parameters.

If the parameters of the equation are not piece-wise constant, the previously mentioned method for selecting the sixth point by Fogelson and Keener \cite{fogelsonkeener00} proves challenging to integrate with existing linear solvers such as the multigrid method.
However Li and Ito \cite{liito01} have developed a method that generates a diagonally dominant coefficient matrix with symmetric negative definite part.
These properties allow it to be used with the modified multigrid method as shown by Adams and Li \cite{adamsli02}.

Although these approaches allow fast linear solvers to be used, they do not attack the primary problem which is that the system generated by the IIM is not symmetric and is not positive definite.
The work of Berthelson \cite{berthelson04} and more recently Zhao et. al. \cite{zhaohouli12} aims to solve this by using standard finite difference coefficients at irregular points and using the correction term only to compensate for the discontinuity.
This approach is promising, and has shown improved accuracy over the original formulation of the IIM and other modifications to the IIM.

These techniques have been extended to three dimensions eg. \cite{dengitoli03}, \cite{dumettkeener03}.

\subsection{Parabolic problems}
The IIM can also be used to solve parabolic problems such as the 2D heat equation
\begin{equation}
    \frac{\partial u}{\partial t} = \kappa \nabla^2 u.
\end{equation}
Similar issues that were discussed in terms of elliptic problems in section \ref{ellipticProblems} occur when solving 2D parabolic problems. 
However there is added complexity in that since there is a time dependence there may now be a discontinuity in $\left[\frac{\partial u}{\partial t}\right]$ over the interface and this must be corrected for when choosing our $\gamma$ values and correction terms.
This is achieved using a similar process of Taylor series expansion as found in section \ref{OneD} along with derivatives of the boundary condition with respect to $t$.

An inherent problem with parabolic problems that is not present when solving elliptic problems is that unconditionally stable implicit methods require the solution of extremely large implicit systems in two and higher dimensions.
One solution to this are so called 'operator splitting' methods which split each time-step into two half steps with one spatial coordinate implicit and one explicit per time-step.
A popular splitting method is the ADI method, and this has been adapted for use with the IIM in Li \cite{li94} and Li and Mayo \cite{limayo93}.
This was achieved by splitting the correction term $C_j$ into x and y components in order to match the splitting of operators employed in the ADI method.

Extra complexity associated with time dependent problems is that the interface may be moving with respect to the domain.
These problems are relatively easy to solve when the position of the interface is prescribed, but become significantly more difficult when the position of the interface must be solved for along with solving the parabolic system, such as for the 'Stefan problem' of a melting crystal.

There are a number of methods that may be employed to find the position of the boundary or interface.
The commonality between all of these methods is that they first use the IIM to solve the equations governing the motion of the interface and then employ some sort of evolution scheme to evolve the position of the boundary between time-steps.

The simplest of these methods is the front tracking method.
This method employs a number of control points which define the position of the boundary and are evolved at each time step according to the local properties of the system (eg. velocity).
A more complex method is the level set method, which uses the zero level set of a level set function to track the interface.
The time derivative of this function can be coupled with the local velocity and then the position of the boundary can be solved for.
Level set methods are generally more robust than front tracking methods because they are able to model processes where there is merging or splitting of boundaries.
The level set method has been used in conjunction with the IIM to solve the aforementioned Stefan problem, as well as the Hele-Shaw flow \cite{liito06}.


\subsection{Hyperbolic problems}
\label{hyper}
The IIM can be used to solve hyperbolic problems such as the 2D wave equation
\begin{equation}
    \frac{\partial^2 u}{\partial t^2} = c^2 \nabla^2 u.
\end{equation}

The IIM for hyperbolic equations has been developed by Zhang \cite{zhang96} and Zhang and Leveque \cite{zhangleveque97} relatively early in the history of the IIM.
The techniques from section \ref{OneD} are again valid for use in higher dimensional hyperbolic systems, however there are some modifications due to the necessity of different discretisation schemes.
Zhang and Leveque \cite{zhangleveque97} employ both the second order Lax-Wendroff scheme and more accurate wave propagation algorithms provided by the CLAWPACK library \cite{leveque93}, although the latter is based on a finite volume discretisation.

In contrast to elliptic and parabolic PDEs, little work has been done on the IIM for hyperbolic PDEs.
A possible explanation for this is that the point of the IIM is to only differ from standard finite difference methods where necessary ie. at interfaces.
Thus, the two possible complications with modifying finite difference schemes at interfaces (namely movement of the interface and interfaces in 2D and higher dimensional problems) have already been dealt with when developing the IIM for parabolic and elliptic equations.

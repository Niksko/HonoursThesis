\section{The IIM for a simple 1D interface problem}
\label{OneD}

The basic outline for solving a problem using the IIM in 1D is to:

\begin{itemize}
    \item Determine jump conditions
    \item Discretise the equation using a particular finite difference method
    \item Separate domain into regular and irregular points
    \item Set up and solve a system that minimizes the truncation error at the irregular points
    \item Solve the modified finite difference equations
\end{itemize}

As an illustration, we will solve a second order linear constant coefficient ODE

\begin{equation}
    a\frac{d^2u}{dx^2} + b\frac{du}{dx} + cu = 0, 
\end{equation}

with $0\leq x \leq 1$, $a$, $b$ and $c$ arbitrary constants, and $u(0) = u_0$, $u(1) = u_1$.

For this problem we will specify  particular jump conditions, but in general the jump conditions might be determined from the physical processes that causes the discontinuities in a problem.

A jump condition defines the relationship between the function (and its derivatives) on either sides of the discontinuity (referred to as the interface). If the discontinuity occurs at $x = \alpha$ then we represent the jump condition for $u(x)$ by $[u]_\alpha$ where

\begin{equation}
    [u]_\alpha := u^+(\alpha) - u^-(\alpha),
\end{equation}

with

\begin{align}
    u^+(\alpha) := \lim_{x \to \alpha^+} u(x), \nonumber \\
    u^-(\alpha) := \lim_{x \to \alpha^-} u(x) \nonumber,
\end{align}

and similarly for $[\frac{du}{dx}]_\alpha$ and $[\frac{d^2u}{dx^2}]_\alpha$.
These conditions will be used to correct the finite difference equations near the interface to preserve the accuracy of the finite difference method.

We can set up a grid of $x$ values in the usual way

\begin{equation}
    x_i = \Delta xi, \qquad i=0,1,2,\ldots,N \nonumber
\end{equation}
for some number of points $N$ with 
\begin{equation}
    \Delta x = \frac{1}{N} \nonumber.
\end{equation}

Note that the IIM works equally well regardless of whether the interface at $x = \alpha$ lies on a gridpoint.
This feature is particularly useful when considering problems where the interface is moving, as it means creating a new mesh at each timestep is not necessary.

In order to maintain accuracy with the least amount of extra computation and complication, the finite difference scheme remains unmodified for points that lie away from the interface.
The points not immediately adjacent to the interface are referred to as {\bf regular} points, while the two points (or one point if the interface lies on a grid-point) closest to the interface are referred to as {\bf irregular} points.
If we take 
\begin{equation}
    x_j, x_{j+1} : x_j \leq \alpha \leq x_{j+1} \nonumber
\end{equation}
to be the irregular points, then 
\begin{equation}
    x_i : i \neq j, j+1
\end{equation}
are the regular points.
If we approximate the value of $u(x_i)$ by $u_i$ we can obtain the familiar three-point centered difference scheme for the regular points

\begin{equation}
    a\frac{u_{i+1} - 2u_i +u_{i-1}}{h^2} + b\frac{u_{i+1} - u_{i-1}}{2h} + cu_i = 0.
\end{equation}
If we collect terms based on $u$ values this becomes
\begin{equation}
    u_{i+1}\left(\frac{a}{h^2} + \frac{b}{2h}\right) + u_i\left(\frac{-2a}{h^2} + c\right) + u_{i-1}\left(\frac{a}{h^2} - \frac{b}{2h}\right) = 0.
\end{equation}

For irregular points, we must generate finite difference coefficients so as to minimize the truncation error at the interface.
The general form of the finite difference equation is

\begin{equation} \label{eq:gammaCoefficients}
    \gamma_{l,1}u_{l-1} + \gamma_{l,2}u_l + \gamma_{l,3}u_{l+1} = C_l, \qquad l = j, j+1,
\end{equation}
with $\gamma_{l,1}$,$\gamma_{l,2}$,$\gamma_{l,3}$,$C_l$ constants to be determined.

We determine the $\gamma_{l,m}$ so as to ensure that the truncation error $T_l$ at the point $x = \alpha$ matches the order of the finite difference method used.

The truncation error is given by
\begin{equation} \label{eq:truncationError}
    T_l= \gamma_{l,1}u(x_{l-1}) + \gamma_{l,2}u(x_l) + \gamma_{l,3}u(x_{l+1}) - C_l - a\frac{d^2u}{dx^2} - b\frac{du}{dx} - cu
\end{equation}
for $l = j, j+1$.

In order to evaluate this expression, it is necessary to take Taylor series expansions of each of the approximate terms and expand them around the point $x = \alpha$. This gives the following expressions:
\begin{equation} \label{eq:taylorExpansions}
\begin{split}
    u(x_{j-1}) &  =  u^-(\alpha) + (x_{j-1} - \alpha)\frac{du}{dx}^-(\alpha) + \frac{1}{2}(x_{j-1} - \alpha)^2\frac{d^2u}{dx^2}^-(\alpha) + O(\Delta x^3), \\
    u(x_{j}) & =  u^-(\alpha) + (x_{j} - \alpha)\frac{du}{dx}^-(\alpha) + \frac{1}{2}(x_{j}-\alpha)^2\frac{d^2u}{dx^2}^-(\alpha) + O(\Delta x^3), \\
    u(x_{j+1}) & =  u^+(\alpha) + (x_{j+1} - \alpha)\frac{du}{dx}^+(\alpha) + \frac{1}{2}(x_{j+1}-\alpha)^2\frac{d^2u}{dx^2}^+(\alpha) + O(\Delta x^3).
\end{split}
\end{equation}


But using our jump conditions we can express $u(x_{j+1})$ strictly in terms from the left hand side of the interface.
Substituting this equation back into the equation for the truncation error and collecting like terms we can obtain and solve a linear system of equations for $\gamma$ that ensures the desired accuracy at $x=\alpha$.
\begin{equation}
    \left\{ \begin{aligned}
        \gamma_{j,1} + \gamma_{j,2} + \gamma_{j,3} & = c \\
        \gamma_{j,1}(x_{j-1}-\alpha) + \gamma_{j,2}(x_j - \alpha) + \gamma_{j,3}(x_{j+1} - \alpha) & = b \\
        \frac{\gamma_{j,1}}{2}(x_{j-1}-\alpha)^2 + \frac{\gamma_{j,2}}{2}(x_j - \alpha)^2 + \frac{\gamma_{j,3}}{2}(x_{j+1} - \alpha)^2 & = a.
    \end{aligned} \right. 
\end{equation}
We also require that 
\begin{equation}
    C_j = \gamma_{j,3}\left([u]_\alpha + (x_{j+1} - \alpha)\left[\frac{du}{dx}\right]_\alpha + \frac{1}{2}(x_{j+1} - \alpha)^2\left[\frac{d^2u}{dx^2}\right]_\alpha\right)
\end{equation}
to give $T_j = O(\Delta x^3)$.

The derivation for the difference equation at $x = x_{j+1}$ is identical except now there are two points that fall to the right of the interface.
Following the same process we see that the only difference is the $C_{j+1}$ term which is now
\begin{align}
    C_{j+1} & = \gamma_{j,2}\left([u]_\alpha + (x_{j+1} - \alpha)\left[\frac{du}{dx}\right]_\alpha + \frac{1}{2}(x_{j+1} - \alpha)^2\left[\frac{d^2u}{dx^2}\right]_\alpha\right) \\ \nonumber
    & + \gamma_{j,3}\left([u]_\alpha + (x_{j+2} - \alpha)\left[\frac{du}{dx}\right]_\alpha + \frac{1}{2}(x_{j+2} - \alpha)^2\left[\frac{d^2u}{dx^2}\right]_\alpha\right) .
\end{align}

We now have a complete system of equations which when solved produces a second order accurate solution to our original problem.

Although here we have considered a fairly simple problem as an illustration, the IIM can be completely generalized for non-constant coefficients that may include discontinuities and for higher orders with little difficulty.
In addition, the IIM is easily implemented for higher order finite difference schemes \cite{liito06}.
However, in order to keep problems simple, this project only considers first and second order finite difference schemes.

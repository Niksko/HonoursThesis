\section{The Russell and Wang method}

The final method is due to Russell and Wang \cite{russellwang03}.
This method does not alter the coefficients of the difference equation at all, but instead adds constant correction terms based off of the jump conditions.

As in the previous two sections, we are solving the heat equation with the conditions given in section \ref{liIto} using the FTCS method.

As in those sections, our difference equation at points away from the interface is

\begin{equation}
    \frac{u_i^{k+1} - u_i^k}{\Delta t} = \kappa \frac{u_{i-1}^k - 2 u_i^k + u_{i+1}^k}{(\Delta x)^2}
\end{equation}

However this difference equation is not valid for points near the interface.

In order to compute the values near the interface $u_j^k$ and $u_{j+1}^k$, observe that if $\overset{\sim}{u}_{j+1}^k$ is the value obtained by solving the system using values from the left of the interface as if there were no interface and corresponding jumps then

\begin{equation}
    u_{j+1}^k = \overset{\sim}{u}_{j+1}^k - [u] - (x_{j+1} - \alpha)\left[\frac{\partial u}{\partial x}\right] - \frac{1}{2}(x_{j+1} - \alpha)^2\left[\frac{\partial^2 u}{\partial x^2}\right] + \ldots
\end{equation}

and similarly for the point at $x_j$.
Hence, we can effectively solve the system by adding the extra terms in the above expression to the RHS of the difference equation.

Similarly, to correct for discontinuities in time we can use an identical derivation to find that

\begin{equation}
    u_j^{k+1} = \overset{\sim}{u}_j^{k+1} - \operatorname{sgn}\left(\frac{d \alpha}{d t}\right)\left\{[u] + (t_{k+1} - \tau)\left[\frac{\partial u}{\partial t}\right] + \frac{1}{2}(t_{k+1} - \tau)^2 \left[\frac{\partial^2 u}{\partial t^2}\right] + \ldots \right\}
\end{equation}

and similarly for $u_j^k$.
